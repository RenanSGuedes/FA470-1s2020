\documentclass[a4paper,12pt]{article}
\usepackage[top=2cm,right=2cm,bottom=2cm,left=2cm]{geometry}
\usepackage[portuguese]{babel}
\usepackage[utf8]{inputenc}
\usepackage[T1]{fontenc}
\usepackage{amsfonts,amsmath,amssymb}
\usepackage{siunitx}
\usepackage{hyperref}
\usepackage{graphicx}
%\usepackage{xwatermark}
\usepackage{xcolor}
%\newwatermark[allpages,scale=6,color=red!15,angle=60,xpos=-60pt,ypos=30pt]{FA47

\begin{document}
	\begin{center}
		\begin{huge}
			FA470 - Dinâmica de Corpos Rígidos\\\vspace{1cm}
		\end{huge}
		\begin{large}
			Professor William Martins Vicente\\\vspace{.5cm}
			\href{https://github.com/renanGuedes10/}{PAD Renan da Silva Guedes}\\\vspace{1cm}
		\end{large}
		\begin{Large}
			R. C. Hibbeler, Dinâmica. Mecânica Para Engenharia, Pearson; Edição: 12$^{\,\underline{\text{a}}}$, 2010
		\end{Large}\\\vspace{1cm}
		\begin{large}
			\textbf{Capítulo 13}
		\end{large}
	\end{center} 

	\begin{enumerate}
		\item Uma mola de rigidez $k=\SI{500}{\newton/\meter}$ está montada contra o bloco de \SI{10}{\kilogram}. Se o bloco está sujeito à força $F=\SI{500}{\newton}$, determine a sua velocidade em $s=\SI{0.5}{\meter}$. Quando $s=0$, o bloco está suspenso e  a mola está descomprimida. A superfície de contato é lisa. 
		
		$\textbf{\text{Resposta}}\Rightarrow v=\SI{5.24}{\meter/\second}$
	
		\item Se o bloco $A$ de \SI{5}{\kilogram} escorrega para baixo no plano inclinado com uma velocidade constante quando $\theta=30^{\circ}$, determine a aceleração do bloco quando $\theta=45^{\circ}$
		
		$\textbf{\text{Resposta}}\Rightarrow a=\SI{}{\meter/\second}$
		
		\item Determine a tração no arame $CD$ logo após o arame $AB$ ser cortado. A pequena esfera tem massa $m$.
	
		$\textbf{\text{Resposta}}\Rightarrow T_{CD}=mg\sin\theta$
		
		\item A caixa tem massa de \SI{80}{\kilogram} e está sendo puxada por uma corrente que está sempre direcionada a $20^{\circ}$ da horizontal, como mostrado. Determine a aceleração da caixa em $t=\SI{2}{\second}$ se o coeficiente de atrito estático $\mu_{s}=0.4$, o coeficiente de atrito cinético $\mu_{k}=0.3$, e a força de reboque é $P=(90\,t^{2})$, onde $t$ é dado em segundos.

		$\textbf{\text{Resposta}}\Rightarrow a=\SI{1.75}{\meter/\second^{2}}$
		
		\item Determine a velocidade máxima que o jipe pode se mover sobre o cume do monte sem perder o contato com a estrada.
		
		Um acrobata tem peso de \SI{750}{\newton} ($m\approx\SI{75}{\kilogram}$) e está sentado em uma cadeira que está fixa no topo de um mastro, como mostrado. Se por um acionamento mecânico o mastro gira para baixo com uma razão constante a partir de $\theta=0^{\circ}$ de tal maneira que o centro de massa $G$ do acrobata mantenha velocidade constante $v_{a}=\SI{3}{\meter/\second}$, determine o ângulo $\theta$ no qual ele começa a "voar"\,para fora da cadeira. Despreze o atrito e suponha que a distância do axial $O$ a $G$ é $\rho=\SI{4.5}{\meter}$
		
		\item Um carro de \SI{0.8}{\mega\g} desloca-se sobre um monte com o formato de parábola. Quando o carro está no ponto $A$, ele está se deslocando a $\SI{9}{\meter/\second}$ e aumentando sua velocidade em $\SI{3}{\meter/\second^{2}}$. Determine a força normal resultante e a força de atrito resultantes que todas as rodas do carro exercem sobre a estrada neste instante. Despreze a dimensão do carro.
		
		\textbf{Resposta}
		$
		\begin{cases}
		\theta=\SI{-26.57}{^{\circ}}\\
		\rho=\SI{223.61}{\meter}\\
		F_{t}=\SI{1.11}{\kilo\newton}\\
		N=\SI{6.73}{\kilo\newton}
		\end{cases}
		$
	\end{enumerate}













\end{document}