\documentclass[a4paper,12pt]{article}
\usepackage[top=2cm,right=2cm,bottom=2cm,left=2cm]{geometry}
\usepackage[portuguese]{babel}
\usepackage[utf8]{inputenc}
\usepackage[T1]{fontenc}
\usepackage{amsfonts,amsmath,amssymb}
\usepackage{siunitx}
\usepackage{hyperref}
\usepackage{graphicx}
%\usepackage{xwatermark}
\usepackage{xcolor}
%\newwatermark[allpages,scale=6,color=red!15,angle=60,xpos=-60pt,ypos=30pt]{FA470}


\newcommand{\solv}[1]{$\textbf{\text{Resolução}\Rightarrow #1$}

\begin{document}
	\begin{center}
		\begin{huge}
			FA470 - Dinâmica de Corpos Rígidos\\\vspace{1cm}
		\end{huge}
		\begin{large}
			Professor William Martins Vicente\\\vspace{.5cm}
			\href{https://github.com/renanGuedes10/}{PAD Renan da Silva Guedes}\\\vspace{1cm}
		\end{large}
		\begin{Large}
			R. C. Hibbeler, Dinâmica. Mecânica Para Engenharia, Pearson; Edição: 12$^{\,\underline{\text{a}}}$, 2010
		\end{Large}\\\vspace{1cm}
		\begin{large}
			\textbf{Capítulo 13}
		\end{large}
	\end{center} 

	\begin{enumerate}
		\item Uma mola de rigidez $k=\SI{500}{\newton/\meter}$ está montada contra o bloco de \SI{10}{\kilogram}. Se o bloco está sujeito à força $F=\SI{500}{\newton}$, determine a sua velocidade em $s=\SI{0.5}{\meter}$. Quando $s=0$, o bloco está suspenso e  a mola está descomprimida. A superfície de contato é lisa. 
		
		\solv{4}
	\end{enumerate}













\end{document}