\documentclass[a4paper, 12pt]{article}
\usepackage[top=2cm,left=2cm,bottom=2cm,right=2cm]{geometry}
\usepackage{amsmath}
\usepackage{amsfonts}
\usepackage{amssymb}
\usepackage[utf8]{inputenc}
\usepackage[T1]{fontenc}
\usepackage{xcolor}

\begin{document}
	\begin{enumerate}
		\item O modelo de agricultura do camponês europeu até meados da década de 60, período inicial da Revolução Verde, foi caracterizado pelo maior apreço na preservação do solo. Dessa forma, os cultivares eram incrementados dispondo de práticas próximas ao conservacionismo. Um exemplo destas está na utilização da rotação de culturas. Através dela, era possível dirimir os danos estruturais ao solo com base na maior diversidade de espécies no plantio. Todavia, a partir da modernização agrícola promovida pelo modelo euro-americano, mudanças ocorreram. Os imigrantes europeus que vieram ao território americano ao dispor de imensa quantidade de terras acompanhada do elevado potencial  produtivo passaram a adotar um modelo de agricultura especulativa e monocultora. Ou seja, nota-se uma transição da concepção conservacionista até então praticada para uma metodologia baseada num sistema predatório. Porém, grande parte dos camponeses europeus não imigrantes mantiveram a prática convencional da rotação na região.
		
		\item Tendo em vista a referida afirmação, concordo com seu conteúdo. A especialização regional de fato é prejudicial ao pequeno produtor. Ao estabelecer um modelo de agricultura vários parâmetros são levados em consideração para melhorar a produtividade do cultivar. Porém, em muitos casos cabe ao produtor adequar seus artifícios de produção de modo a atender os critérios associados à área plantada e o clima presente na localidade. Contudo, essas ações podem demandar aparatos e ferramentas que são inacessíveis ao pequeno produtor. Nesse caso, por ser menos favorecido e não ter os recursos para a especialização em sua região ele passa a ser vítima do monopólio do seu setor de produção e, consequentemente, perde para as elites agrárias que dispõem de grandes parcelas de terra aliadas a uma produção monocultora que explora as características favoráveis ao cultivar. Ou seja, há uma desigualdade na escala do sistema de produção e nos recursos que permitem os produtores lidar com os empecilhos no plantio.
		
		\item Barker (1960) ao expor que a mecanização é viável e de baixo custo sistemas monocultores foi um pouco restrito. Atualmente existe o aparato técnico adequado para a fabricação de maquinário multitarefas no campo. Aliado a esse aparato, temos que o custo de produção dessas ferramentas auxiliares não é elevado e demonstra ser acessível de ser implementado. A principal questão está ligada à demanda desse tipo de equipamento. Ou seja, a tendência de mercado não exige o implemento de máquinas \textit{multi-purpose}, mas sim, seria viável a aquisição delas e seu uso de modo a maximizar a eficiência em sistemas de produção de maior complexidade.
	\end{enumerate}
\end{document}