\documentclass[a4paper, 12pt]{article}
\usepackage[top=1cm,left=2cm,bottom=2cm,right=2cm]{geometry}
\usepackage[utf8]{inputenc}
\usepackage[T1]{fontenc}
\usepackage[portuguese]{babel}
\usepackage{amsmath, amsfonts, amssymb}
\usepackage{graphicx}
\usepackage{xcolor}
\usepackage{siunitx}
\usepackage[version=4]{mhchem}
\usepackage{hyperref}

\newcommand{\Sol}{\textbf{Resolução:}}
\newcommand{\tmass}{\text{m}_{\text{T}}}
\newcommand{\wmass}{\text{m}_{\ce{H2O}}}

\newcommand{\smass}{\text{m}_{\text{\text{s}}}}
\newcommand{\ubu}{\text{u}_{\text{bu}}}
\newcommand{\ubs}{\text{u}_{\text{bs}}}
\newcommand{\ur}{\text{UR}}
\newcommand{\kk}{\text{K}}
\newcommand{\tpt}{\text{T}}

\newcommand{\ue}{\text{U}_{\text{e}}}

\title{Resolução da lista 1}
\author{Renan da Silva Guedes}

\begin{document}
	\maketitle
	
	\begin{enumerate}
		\item Dentre os métodos estudados em aula, quais os considerados diretos e os indiretos? Em que se baseia esta classificação? 
		
		\item Um produtor colheu 1000 toneladas de arroz com umidade inicial de 22\%.
		
		\begin{enumerate}
			\item Qual a massa de produto seco?\\
			
			\begin{itemize}
				\item $\tmass=\SI{1000}{\tonne}$
				\item $=\SI{22}{\percent}$
				\item $r$
			\end{itemize}
		
			\begin{eqnarray}
				\ubu&=&\dfrac{\wmass}{\tmass}\\
				0.22&=&\dfrac{\wmass}{1000}\Rightarrow\\
				\Rightarrow\wmass&=&\SI{220}{\tonne}
			\end{eqnarray}
			
			\begin{eqnarray}
				\tmass&=&\wmass+\smass\Rightarrow\\
				\Rightarrow 1000&=&220+\smass\Rightarrow\\
				\Rightarrow\smass&=&\textcolor{red}{\SI{780}{\tonne}}
			\end{eqnarray}
	
			\item Qual o teor de umidade em base seca?
			
			\begin{eqnarray}
				\ubs&=&\dfrac{\wmass}{\smass}\Rightarrow\\
				&=&\dfrac{220}{780}\\
				&=&\textcolor{red}{\SI{28.2}{\percent}}
			\end{eqnarray}
			
			\item Para secar os grãos para 10\% de umidade, quanto de água deve ser removida?
			
			\begin{equation}
				22\% \rightarrow 10\%
			\end{equation}
			
			Retirar $x\,\SI{}{\tonne}$ de água
			
			\begin{eqnarray}
				\ubu&=&\dfrac{\wmass}{\tmass}\Rightarrow\\
				\Rightarrow 0.10&=&\dfrac{\wmass-x}{\smass+\wmass-x}\Rightarrow\\
				\Rightarrow 0.10&=&\dfrac{220-x}{1000-x}\Rightarrow\\
				\Rightarrow 100-0.1\,x&=&220-x\Rightarrow\\
				\Rightarrow 0.9\,x&=&120\Rightarrow\\
				\Rightarrow x&=&\SI{133.3333}{\tonne}	
			\end{eqnarray}
		\end{enumerate}
		
		\item Cinco toneladas de Soja estavam com 13\% de umidade e foram levadas para um ambiente com 90\% de umidade relativa e \SI{25}{\celsius}. Dados: $\text{n}=1,52$ e $\text{K}=3,20\cdot 10^{-5}$
		
		\begin{enumerate}
			
			\item Qual a nova umidade da soja?\\
			
			\begin{minipage}{.25\linewidth}
				\begin{itemize}
					\item $\tmass=\SI{5}{\tonne}$
					\item $\ubu=\SI{13}{\percent}$
					\item $\ur=\SI{90}{\percent}$
					\item $\tpt=\SI{25}{\celsius}$
					\item $\text{n}=1.52$
					\item $\kk=3.2\cdot 10^{-5}$
				\end{itemize}
			\end{minipage}
			\begin{minipage}{.75\linewidth}
				
				\underline{Conversão de temperatura}
				
				\begin{equation}
					\tpt(\text{Ra})=(25+273.15)\cdot\frac{9}{5}
				\end{equation}
				\underline{Equação de Henderson}
				
				\begin{eqnarray}
					1-\ur&=&\exp(-\kk\,\tpt\,\ue^{\text{n}})\Rightarrow\\
					\Rightarrow 1-.0.9&=&\exp(-3.2\cdot 10^{-5}\cdot\tpt \cdot\ue^{1.25})\Rightarrow\\
					\Rightarrow\ue&=&\SI{25.0944}{\percent}				
				\end{eqnarray}
			\end{minipage}
		\end{enumerate}
	
		\item Dez toneladas de milho estavam em equilíbrio a uma umidade relativa de 80\% e temperatura de \SI{30}{\celsius}. Todo este produto foi levado a um ambiente onde a umidade de equilíbrio final foi de 10\%. Sabendo-se que o \SI{}{\kilogram} de milho custa R\$1,50 qual o ganho ou perda causado pela mudança de condições. Dados: $\text{n}=1.90$ e $\kk=1.1\cdot10^{-5}$
	\end{enumerate}





















\end{document}
